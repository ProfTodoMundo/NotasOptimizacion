
\documentclass{article}
\usepackage{amssymb}

%%%%%%%%%%%%%%%%%%%%%%%%%%%%%%%%%%%%%%%%%%%%%%%%%%%%%%%%%%%%%%%%%%%%%%%%%%%%%%%%%%%%%%%%%%%%%%%%%%%%%%%%%%%%%%%%%%%%%%%%%%%%%%%%%%%%%%%%%%%%%%%%%%%%%%%%%%%%%%%%%%%%%%%%%%%%%%%%%%%%%%%%%%%%%%%%%%%%%%%%%%%%%%%%%%%%%%%%%%%%%%%%%%%%
%TCIDATA{OutputFilter=LATEX.DLL}
%TCIDATA{Version=5.00.0.2552}
%TCIDATA{<META NAME="SaveForMode" CONTENT="1">}
%TCIDATA{Created=Monday, December 12, 2005 09:01:29}
%TCIDATA{LastRevised=Monday, December 19, 2005 07:44:11}
%TCIDATA{<META NAME="GraphicsSave" CONTENT="32">}
%TCIDATA{<META NAME="DocumentShell" CONTENT="Standard LaTeX\Blank - Standard LaTeX Article">}
%TCIDATA{CSTFile=40 LaTeX article.cst}
%TCIDATA{ComputeDefs=
%$q(x)=q(x_{k}+p)=\frac{1}{2}p^{T}Gp$
%$f_{j-1}=d^{T}x(t_{j-1})$
%}


\newtheorem{theorem}{Theorem}
\newtheorem{acknowledgement}[theorem]{Acknowledgement}
\newtheorem{algorithm}[theorem]{Algorithm}
\newtheorem{axiom}[theorem]{Axiom}
\newtheorem{case}[theorem]{Case}
\newtheorem{claim}[theorem]{Claim}
\newtheorem{conclusion}[theorem]{Conclusion}
\newtheorem{condition}[theorem]{Condition}
\newtheorem{conjecture}[theorem]{Conjecture}
\newtheorem{corollary}[theorem]{Corollary}
\newtheorem{criterion}[theorem]{Criterion}
\newtheorem{definition}[theorem]{Definition}
\newtheorem{example}[theorem]{Example}
\newtheorem{exercise}[theorem]{Exercise}
\newtheorem{lemma}[theorem]{Lemma}
\newtheorem{notation}[theorem]{Notation}
\newtheorem{problem}[theorem]{Problem}
\newtheorem{proposition}[theorem]{Proposition}
\newtheorem{remark}[theorem]{Remark}
\newtheorem{solution}[theorem]{Solution}
\newtheorem{summary}[theorem]{Summary}
\newenvironment{proof}[1][Proof]{\noindent\textbf{#1.} }{\ \rule{0.5em}{0.5em}}
\input{tcilatex}

\begin{document}


PROGRAMACI\'{O}N CUADR\'{A}TICA

\bigskip

Los problemas de programaci\'{o}n cuadr\'{a}tica (QP) son aquellos que
buscan minimizar una funci\'{o}n objetivo cuadr\'{a}tica sujeta a
restricciones lineales. El problema general de programaci\'{o}n cuadr\'{a}%
tica se puede plantear de la siguiente forma (la numeraci\'{o}n de las f\'{o}%
rmulas corresponde a la del libro "Numerical Optimization" de J. Nocedal y
S. Wright, disponible en la biblioteca CIMAT):

\bigskip

$\min_{x}$ $q(x)=\frac{1}{2}x^{T}Gx+x^{T}d\qquad \qquad \qquad \qquad \qquad
\ \ \ \ (16.1a)$

sujeto a$\qquad a_{i}^{T}x=b_{i},\qquad i\in \varepsilon ,\qquad \qquad
\qquad \qquad (16.1b)$

\qquad \qquad $\qquad a_{i}^{T}x\geq b_{i},\qquad i\in I,\qquad \qquad
\qquad \qquad \ (16.1c)$

\bigskip

donde G es una matriz sim\'{e}trica de $n$ x $n$, $\varepsilon $ e $I$ son
conjuntos finitos de \'{\i}ndices, $d,xy$ \{$a_{i}$\}, $i$ $\in \varepsilon $
$\cup $ $I,$son vectores con $n$ elementos$.$Si la matriz G es definida
positiva, el problema es convexo y aproximadamente tan f\'{a}cil de resolver
como un problema de programaci\'{o}n cuadr\'{a}tica. De lo contrario el
problema puede presentar m\'{\i}nimos locales y ser mas dif\'{\i}cil de
resolver.

\bigskip

PROGRAMACI\'{O}N CUADR\'{A}TICA CON\ RESTRICCIONES DE IGUALDAD (QPS)

\bigskip

Primero se revisara el caso de (QP) cuando solo hay restricciones de
igualdad presentes. Este caso particular es \'{u}til porque funciona como
auxiliar en casos mas generales donde para resolver un problema mas grande,
se resuelven subproblemas con restricciones de igualdad.

\bigskip

Se supondr\'{a} que el n\'{u}mero de restricciones es m, y se aumira que m $%
\leq $ n, y el problema de programaci\'{o}n cuadr\'{a}tica se establecer\'{a}
como sigue:

\bigskip

$\min_{x}$ $q(x)=\frac{1}{2}x^{T}Gx+x^{T}d\qquad \qquad \qquad \qquad \qquad
\qquad \qquad (16.3a)$

sujeto a$\qquad a^{T}x=b,\qquad \qquad \qquad \qquad \qquad \qquad \qquad
\qquad (16.3b)$

donde $A$ es la matriz Jacobiana de $m$ x $n$ definida como:

\bigskip

$A$=[$a_{i}$]$_{i\in \varepsilon }^{T}.$

\bigskip

Se asumir\'{a} que A es de rango completo y que las restricciones son
consistentes. El Lagragiano y su gradiente ser\'{a}n, respectivamente:

\bigskip

$\tciLaplace (x)=\frac{1}{2}x^{T}Gx+x^{T}d-\lambda ^{T}(Ax-b)$

$\bigtriangledown $\bigskip $\tciLaplace (x)=Gx+d-A^{T}\lambda $

donde $\lambda $ es un vector de multiplicadores de Lagrange. Aplicando las
condiciones de Karush--Kuhn--Tucker (KKT) obtenemos el siguiente sistema:

\bigskip

$\left[ 
\begin{array}{cc}
G & -A^{T} \\ 
A & 0%
\end{array}%
\right] \left[ 
\begin{array}{c}
x^{\ast } \\ 
\lambda ^{\ast }%
\end{array}%
\right] =\left[ 
\begin{array}{c}
-d \\ 
b%
\end{array}%
\right] \qquad \qquad \qquad \qquad \qquad (16.4)$

\bigskip

donde $x^{\ast }$ es una soluci\'{o}n del sistema y $\lambda ^{\ast }$ es el
vector de multiplicadores de Lagrange en la soluci\'{o}n. Ahora, se
descompone el vector $x^{\ast }=x+p$, donde $x$ es alguna estimaci\'{o}n de
la soluci\'{o}n, y $p$ es el paso que se esta buscando para llegar a la
soluci\'{o}n. Introduciendo esta notaci\'{o}n en (16.4) tenemos:

\bigskip

$\left[ 
\begin{array}{cc}
G & -A^{T} \\ 
A & 0%
\end{array}%
\right] \left[ 
\begin{array}{c}
-p \\ 
\lambda ^{\ast }%
\end{array}%
\right] =\left[ 
\begin{array}{c}
g \\ 
c%
\end{array}%
\right] \qquad \qquad \qquad \qquad \qquad (16.5)$

\bigskip

donde:

\bigskip

$c=Ax-b,\qquad \qquad g=d+Gx,\qquad \qquad p=x^{\ast }-x.\qquad (16.6)$

\bigskip

La matriz de (16.5) se conoce como la matriz de Karuxh--Kuhn--Tucker (KKT).
A continuaci\'{o}n se mencionan dos lemas que son importantes para los
cuales la matriz es no singular y la solucion $x^{\ast }$ es global. $Z$
denota la matriz de $n$ x ($n$--$m$) cuyas columnas son una base para el
espacio nulo de A.

\bigskip

Lema 16.1.

Sea A una matriz de rango por renglones y asumiendo que el Hessiano reducido 
$Z^{T}GZ$ es positivo definido. Entonces la matriz KKT:

\bigskip

$K=\left[ 
\begin{array}{cc}
G & -A^{T} \\ 
A & 0%
\end{array}%
\right] \qquad \qquad \qquad \qquad \qquad \qquad \qquad \qquad (16.7)$

\bigskip

es no singular, y hay un \'{u}nico par de vectores $(x^{\ast },\lambda
^{\ast })$ satisfaciendo (16.4).

\bigskip

Lema 16.2.

\bigskip

Sup\'{o}ngase que las condciones del lema 16.1 son satisfechas. Entonces el
vector $x^{\ast }$ que satisface (16.4) es la \'{u}nica soluci\'{o}n global
de (16.3).

\bigskip

Se puede encontrar las demostraciones a estos lemas en el libro "Numerical
Optimization" en el capitulo 16.1.

\bigskip

RESOLVIENDO\ EL\ SISTEMA\ KKT.

\bigskip

Existen varios m\'{e}todos para resolver el sistema KKT (16.4) o el (16.5).
Primero mencionaremos un lema que es \'{u}til para saber que m\'{e}todo
conviene usar.

\bigskip

Lema 16.3.

Sup\'{o}ngase que $A$ tiene rango completo en los renglones y que el
Hessiano reducido $Z^{T}GZ$ es positivo definido. Entonces la matriz KKT
(16.7) tiene $n$ eigen valores positivos, $m$ negativos y ninguno igual a
cero.

\bigskip

Soluci\'{o}n Directa

\bigskip

Una opci\'{o}n es realizar una factorizaci\'{o}n y hacer substituci\'{o}n
hacia adelante y hacia atr\'{a}s. La factorizaci\'{o}n de Cholesky no
funciona ya que la matriz del sistema (16.4-5) es indefinda. Por otro lado,
la eliminaci\'{o}n Gaussiana con pivoteo no toma en cuenta la simetr\'{\i}a
del sistema. Lo mas efectivo es usar una factorizaci\'{o}n sim\'{e}trica
indefinida como se describen en el cap\'{\i}tulo 6 del libro "Numerical
Optimization". Para una matriz sim\'{e}trica K, esta factorizaci\'{o}n tiene
la forma:

\bigskip

$P^{T}KP=LBL^{T}\qquad \qquad \qquad \qquad \qquad \qquad \qquad (16.12)$

\bigskip

donde P es una matriz de permutaci\'{o}n, L es una matriz unitaria inferior
triangular, y B es una matriz diagonal a bloques, con bloques de 1 x 1 o 2 x
2. Las permutaciones sim\'{e}tricas definidas por la matriz P son para
mantener la estabilidad n\'{u}merica y para mantener la dispersidad. Este
tipo de factorizaci\'{o}n es aproximadamente la mitad de una eliminacion
Gaussiana. Para resolver (16.5) calculamos la factorizaci\'{o}n (16.12),
sustituyendo la matriz KKT por K. Entonces realizamos la siguiente secuencia
de operaciones para llegar a la soluci\'{o}n:

\bigskip

resolver $Ly=P^{T}\left[ 
\begin{array}{c}
g \\ 
c%
\end{array}%
\right] $ para obtener $y$;

\bigskip resolver $By^{\symbol{94}}=y$ para obtener $y^{\symbol{94}}$;

resolver $L^{T}y^{-}=y^{\symbol{94}}$ para obtener $y^{-}$;

hacer $\left[ 
\begin{array}{c}
-p \\ 
\lambda ^{\ast }%
\end{array}%
\right] =Py^{-}$.

\bigskip

La mayor\'{\i}a de las operaciones que implica este m\'{e}todo pueden
implementarse de manera muy eficiente. La multiplicaci\'{o}n por la matriz
de permutaci\'{o}n $P$ y su transpuesta, por ejemplo, implica solamente el
reacomodo de renglones o columnas. Sin embargo, puede ser dif\'{\i}cil
definir la matriz de permutaci\'{o}n $P$.

\bigskip

M\'{e}todo del Espacio de Rangos.

\bigskip

En el m\'{e}todo del espacio de rangos, se usa la matriz G para realizar una
eliminaci\'{o}n por bloques en el sistema (16.5). Asumiendo que G es
positiva definida, se multiplica la primera ecuaci\'{o}n en (16.5) por $%
AG^{-1}$ y entonces se extrae la segunda ecuaci\'{o}n para obtener un
sistema linear en el vector $\lambda ^{\ast }$:

\bigskip

$(AG^{-1}A^{T})\lambda ^{\ast }=(AG^{-1}g-c).\qquad \qquad \qquad (16.13)$

\bigskip

Se resuelve este sistema sim\'{e}trico positivo definido para $\lambda
^{\ast }$, y se recupera $p$ de la primera ecuaci\'{o}n en (16.5) resolviendo

\bigskip

$Gp=A^{T}\lambda ^{\ast }-g\qquad \qquad \qquad \qquad \qquad \qquad (16.4)$

\bigskip

Este m\'{e}todo requiere realizar operaciones con $G^{-1}$, as\'{\i} como
calcular la factorizaci\'{o}n de la matriz de $m$ x $m$, $AG^{-1}A^{T}$. Por
lo tanto es mas \'{u}til cuando:

\bigskip

\begin{itemize}
\item $G$ es bien condicionada y f\'{a}cil de invertir, (duando G es
diagonal o diagonal por bloques).

\item $G^{-1}$ es conocida expl\'{\i}citamente a trav\'{e}z de una formula
de actualizaci\'{o}n de cuasi-Newton o

\item el n\'{u}mero de ecuaciones de igualdad $m$ es peque\~{n}o, de manera
que el n\'{u}mero de resoluciones hacia atr\'{a}s necesario para formar la
matriz $AG^{-1}A^{T}$ no es muy grande.
\end{itemize}

\bigskip

PROBLEMAS CON RESTRICCIONES DE DESIGUALDAD

\bigskip

En el resto del cap\'{\i}tulo se discutir\'{a}n varios algoritmos para
resolver problemas de programaci\'{o}n cuadr\'{a}tica que contienen
restricciones de desigualdad. El m\'{e}todo de conjuntos activos es ya cl%
\'{a}sico y puede ser aplicado tanto en m\'{e}todos convexos como no
convexos. El m\'{e}todo de proyecci\'{o}n de gradiente tratan de acelerar la
soluci\'{o}n, permitiendo cambios r\'{a}pidos en el conjunto activo. Los m%
\'{e}todos de punto interior han demostrado ser efectivos para resolver
problemas cuadr\'{a}ticos convexos grandes.

\bigskip

CONDICIONES DE OPTIMALIDAD PARA PROBLEMAS CON RESTRICIONES DE DESIGUALDAD

\bigskip

Se discutir\'{a}n algunas de las propiedades de los problemas de programaci%
\'{o}n cuadr\'{a}tica con restricciones de desigualdad. Aplicando el
Lagragiano a (16.1), tenemos:

\bigskip

$\tciLaplace (x,\lambda )=\frac{1}{2}x^{T}Gx+x^{T}d-\sum_{i\in I\cup
\varepsilon }\lambda _{i}(a_{i}^{T}x-b_{i})\qquad \qquad \qquad \qquad
(16.24)$

\bigskip

Adem\'{a}s, definimos el conjunto activo $A(x^{\ast })$ en un punto \'{o}%
ptimo $x^{\ast }$ como los \'{\i}ndices de las restricciones en las cuales
la igualdades se cumplen, es decir,

\bigskip

$A(x^{\ast })=\{i\in \varepsilon \cup I:a_{i}^{T}x^{\ast }=b_{i}\}\qquad
\qquad \qquad \qquad \qquad \qquad \qquad (16.25)$

\bigskip

Simplificando las condiciones necesarias de primer orden, incluimos que
cualquier soluci\'{o}n $x^{\ast }$ de (16.1) satisfacen las siguientes
condiciones de primer orden:

\bigskip 

\begin{eqnarray*}
Gx^{\ast }+d-\sum_{i\in A(x^{\ast })}\lambda _{i}^{\ast }a_{i} &=&0,\text{ \
\ \ \ \ \ \ \ \ \ \ \ \ \ \ \ \ \ \ \ \ \ \ \ \ \ \ \ \ \ \ \ \ \ \ \ \ \ \
\ \ \ \ \ \ \ \ \ \ \ \ \ \ \ \ \ \ \ \ \ \ \ \ \ (16.26a)} \\
a_{i}^{T}x^{\ast } &=&b_{i},\text{ para toda }i\in A(x^{\ast }),\text{ \ \ \
\ \ \ \ \ \ \ \ \ \ \ \ \ \ \ \ \ \ \ \ \ \ \ \ (16.26b)} \\
a_{i}^{T}x^{\ast } &\geq &b_{i},\text{ para toda }i\in I\setminus A(x^{\ast
}),\text{ \ \ \ \ \ \ \ \ \ \ \ \ \ \ \ \ \ \ \ \ \ (16.26c)} \\
\lambda _{i}^{\ast } &\geq &0,\text{para toda i}\in I\cap A(x^{\ast }),\text{
\ \ \ \ \ \ \ \ \ \ \ \ \ \ \ \ \ \ \ \ \ \ \ (16.26d)}
\end{eqnarray*}

\bigskip 

Las condiciones suficientes de segundo orden para que $x^{\ast }$ sea un
minimizador local son satisfechas si $Z^{T}GZ$ es positiva definida, donde $Z
$ es  una matriz base del espacio nulo para la matriz Jacobiana de
restriciones activas

\bigskip 

$[a_{i}]_{i\in A(x^{\ast })}^{T}.$

\bigskip 

$x^{\ast }$ es la soluci\'{o}n global cuando las restricciones son de
igualdad. Cuando $G$ no es positiva definida, pueden existir varios m\'{\i}%
nimos locales estrictos en los cuales las condiciones necesarias de segundo
orden son satisfechas. Estos problemas son conocidos como como no "convexos"
o "indefinidos" y pueden causar complicaciones en los algoritmos.

\bigskip 

DEGENERACI\'{O}N

\bigskip 

Otra propiedad que puede causar problemas para algunos algoritmos es la
degeneraci\'{o}n. Esto se refiere a situaciones en las que pueden ocurrir
alguna de estas cosas

\begin{itemize}
\item los gradientes de las restricciones activas $a_{i}$, $i\in A(x^{\ast })
$, son linealmente dependientes en la solucion $x^{\ast }$, o

\item la condici\'{o}n de complementariedad estricta no se cumple, o sea que
el vector de multplicadores de Lagrange en el \'{o}ptimo contiene elementos $%
\lambda _{i}=0$ para algunos \'{\i}ndices $i\in A(x^{\ast })$. 
\end{itemize}

La degeneraci\'{o}n puede traer como consecuencia, dificultad en el c\'{a}%
lculo de la matriz de espacio nulo $Z$, lo que a su vez provoca que la
matriz $AG^{-1}A^{T}$ del m\'{e}todo del espacio de rangos sea singular.
Cuando las restricciones son d\'{e}bilmente activas, puede ser dif\'{\i}cil
para un algoritmo decidir si esas restricciones son activas o no.

\bigskip 

\bigskip M\'{E}TODO DE LOS CONJUNTOS ACTIVOS PARA PROGRAMACI\'{O}N CUADR\'{A}%
TICA

Estos m\'{e}todos son efectivos para problemas peque\~{n}os y medianos.
Asumen inicialmente que $G$ es semidefinida positiva. Escencialmente
trabajan con subconjuntos de restricciones que est\'{a}n activas, y
actualizan este subconjunto hasta dar con la soluci\'{o}n \'{o}ptima. El m%
\'{e}todo simplex es un m\'{e}todo de conjunto activos, que entre sus
caracter\'{\i}sticas esta que la soluci\'{o}n siempre se encuentra en uno de
los v\'{e}rtices del poliotipo que forman las restricciones.\bigskip 

Los tres m\'{e}todos de conjuntos activos que se manejan son:

\begin{itemize}
\item Primal dual-$(x,\lambda )$

\item Primal-$(x)$

\item Duales-$(\lambda )$
\end{itemize}

Usualmente se asume un punto factible $x_{0}$ y se contin\'{u}a con una
secuencia que permanece factible. Cada paso consiste en resolver un
subproblema cuadr\'{a}tico con restricciones de igualdad. A cada conjunto de
restricciones se le denomina "working set" o conjunto activo y es denotado
en la $k-$\'{e}sima iteraci\'{o}n como $W_{k}$ y contiene a todas las
restricciones de igualdad y algunas de desigualdad (activas). 

\bigskip 

Dada $x_{k}$ y $W_{k}$ revisamos si $x_{k}$ minimiza $q(x)=x^{T}Gx+x^{T}d$
en el subespacio definido por $W_{k}$, si no, calculamos un paso $p$
resolviendo el subproblema con restricciones de igualdad. Definimos:

\bigskip 

$p=x-x_{k}$, \ \ \ \ \ \ \ \ \ $q_{k}=Gx_{k}+d$

\bigskip 

luego

\bigskip 

$q(x)=q(x_{k}+p)=\frac{1}{2}p^{T}Gp+g_{k}^{T}p+c$

\bigskip 

donde $c=\frac{1}{2}x_{k}^{T}Gx_{k}+d^{T}x_{k}$ es una constante, y por lo
tanto podemos desecharla sin alterar el m\'{\i}nimo del problema. Se escribe
el subproblema de programaci\'{o}n cuadr\'{a}tica a resolver en la $k$-\'{e}%
sima iteraci\'{o}n como:

\bigskip 

\begin{eqnarray*}
&&\min_{p}\frac{1}{2}p^{T}Gpg_{k}^{T}p \\
\text{sujeto a }a_{i}^{T}p &=&0\text{ \ para toda }i\in W_{k}
\end{eqnarray*}

$\bigskip $

Y se le llama $p_{k}$ a la soluci\'{o}n de este subproblema. Sup\'{o}ngase
que $p_{k}\neq 0$. Debe decidirse que tan lejos se avanza en esta direcci%
\'{o}n. Si $x_{k}+p_{k}$ es factible para todas las restricciones, entonces
la variable $x_{k+1}$ ser\'{a} igual a esta suma. De lo contrario, el valor
de $x_{k+1}$ ser\'{a} igual a:

\bigskip 

\[
x_{k+1}=x_{k}+\alpha _{k}p_{k}
\]

donde $\alpha $ es el valor mas alto en  $[0,1)$ que no viola ninguna
restricci\'{o}n. Para cualquier valor de alfa, las restricciones del
conjunto activo se cumplen, por lo tanto, para todas las restricciones que
no estan en el conjunto activo se debe cumplir que $a_{i}^{T}(x_{k}+\alpha
_{k}p_{k})\geq \alpha _{i}^{T}x_{k}\geq b_{i}$. De esto se deriva que para
todas las restricciones fuera del conjunto activo, se debe cumplir que:

\bigskip 

$\alpha _{k}\leq \frac{b_{i}-a_{i}^{T}x_{k}}{a_{i}^{T}p_{k}}$

\bigskip 

Como se busca hacer $\alpha _{k}$ lo m\'{a}s grande posible en el intervalo $%
[0,1]$ sujeto a mantener la factibilidad, se tiene la siguiente definici\'{o}%
n:

\bigskip 

$\alpha _{k}=\min \left( 1,\min_{i\notin W_{k},a_{i}^{T}p_{k\,0}}\frac{%
b_{i}-a_{i}^{T}x_{k}}{a_{i}^{T}p_{k}}\right) \qquad \qquad \qquad (16.29)$

\bigskip 

Se le llama restricci\'{o}n bloqueadora a aquella para la cual el m\'{\i}%
nimo en (16.29) es alcanzado. Si $\alpha =1$, entonces no hay restricciones
bloqueadoras y no se agrega ning\'{u}n nuevo elemento al conjunto activo. Si 
$\alpha \,<1$, el conjunto activo es actualizado agregando una de las
restricciones bloqueadoras. Seguimos iterando de esta manera hasta encontrar
un punto que minimiza la funci\'{o}n objetivo para el conjunto activo
actual. Esto se detecta f\'{a}cilmente, ya que el vector $p_{k}$ ser\'{a}
igual a cero.

Lo que sigue es examinar los multiplicadores de Lagrange del conjunto activo
actual. Si alguno de los multiplicadores de las restricciones de desigualdad
es menor a cero, entonces la condici\'{o}n (16.26d) no se cumple. Se debe
retirar tal restricci\'{o}n del conjunto activo y seguir con el algoritmo.

\bigskip 

A continuaci\'{o}n se presenta un pseudoc\'{o}digo del m\'{e}todo de los
conjuntos activos:

\bigskip 

CONJUNTOS ACTIVOS

Calcular un punto inicial factible $x_{0}$;

Hacer $W_{0}$ un subconjunto de las restricciones activas en $x_{0}$

\textbf{para} $k=0,1,2,...$

\qquad Resolver (16.27) y encontrar $pk$;

\qquad \textbf{si} $p_{k}=0$

\qquad \qquad Calcular los multiplicadores de Lagange $\hat{\lambda}_{i}$
que satisface (16.30),

\qquad \qquad \qquad Hacer $\hat{W}=W_{k}$;

\qquad \qquad \textbf{si} $\hat{\lambda}_{i}\geq 0$ para todo $i\in
W_{k}\cap I$;

\qquad \qquad \qquad DETENER con las soluci\'{o}n $x^{\ast }=x_{k}$;

\qquad \qquad \textbf{De lo contrario}

\qquad \qquad \qquad Hacer $j=\arg \min_{j\in W_{k}\cap i}\hat{\lambda}_{j}$;

\qquad \qquad \qquad $x_{k+1}=x_{k}$; $W_{k+1}\leftarrow W_{k}\diagdown \{j\}
$;

\qquad \textbf{De lo contrario} ($^{\ast }p_{k}\neq 0^{\ast }$)

\qquad \qquad Calcular $\alpha _{k}$ de (16.29);

\qquad \qquad $x_{k+1}\leftarrow x_{k}+\alpha _{k}p_{k}$;

\qquad \qquad \textbf{Si} existen restricciones bloqueadoras

\qquad \qquad \qquad Obtener $W_{k+1}$ agregando una de las restricciones
bloqueadoras a $W_{k+1}$;

\qquad \qquad \textbf{De lo contrario}

\qquad \qquad \qquad $W_{k+1}\leftarrow W_{k}$

\textbf{fin} ($^{\ast }$\textbf{para}$^{\ast }$)

\bigskip M\'{E}TODO DE PROYECCI\'{O}N DE GRADIENTE

\bigskip 

Este m\'{e}todo esta dise\~{n}ado para realizar cambios rapidos en el
conjunto de restricciones activas. Es muy eficiente e problemas de
restricciones de cota y puede aplicarse por igual a problemas convexos y no
convexos. Se mantendr\'{a} el enfoque en problemas del tipo siguiente:

\bigskip 

$\min_{x}q(x)=\frac{1}{2}x^{T}Gx+x^{T}d\qquad \qquad \qquad $(16.43a)

sujeto a $l\leq x\leq u$,\qquad \qquad \qquad \qquad \qquad (16.43b)

\bigskip 

donde G es sim\'{e}trica y $l$ y $u$ son vectores de l\'{\i}mite superior e
inferior en los componentes de $x$., la zona factible tiene forma de caja.
Cada iteraci\'{o}n del m\'{e}todo de proyecci\'{o}n de gradiente consiste en
dos etapas. En la primera se mueve en la direcci\'{o}n de m\'{a}ximo
descenso a partir del punto acutal $x$. Cuando un l\'{\i}mite es alcanzado,
la direcci\'{o}n de b\'{u}squeda es "doblada" de forma que se mantiene
factible. Buscamos a lo largo de este camino a pedazos y localizamos el
primer minimizador local de $q$ al que se llamar\'{a} $x^{c}$, tambi\'{e}n
conocido como punto de Cauchy. En la segunda etapa se "explora" la cara de
la caja donde se encuentra el punto de Cauchy resolviendo un subproblema en
el cual los componenetes activos $x_{i}$ para cada $i\in A(x^{c})$ est\'{a}n
fijos a los valores $x_{i}^{c}$. Se usar\'{a}n super \'{\i}ndices para
indicar el n\'{u}mero de iteraci\'{o}n y sub\'{\i}ndices para indicar
elementos de un vector.

\bigskip 

C\'{A}LCULO\ DEL\ PUNTO\ DE CAUCHY

\bigskip 

La proyecci\'{o}n de un punto arbitrario $x$ dentro de la regi\'{o}n
factible $\Omega $ (16.43b) esta dada por 

\bigskip 

$P(x,l,u)=\left\{ 
\begin{array}{cc}
l_{i} & x_{i}<l_{i} \\ 
x_{i} & l_{i}\leq x_{i}\leq u_{i} \\ 
u_{i} & x_{i}\geq u_{i}%
\end{array}%
\right. $

\bigskip 

para el $i-$\'{e}simo componente de $x_{i}$.

\bigskip 

Luego el camino recto a pedazos, que inicia en $x_{0}$, a lo largo de $-g$,
donde $g=Gx+d$, est\'{a} dado por:

\bigskip 

$x(t)=P(x^{0}-tg,l,u)$

\bigskip 

N\'{o}tese que el problema de cota puede tener a lo m\'{a}s $n$
restricciones activas.

\bigskip 

Ahora se calcula el punto de Cauchy $x^{c}$, el cual se define como el
primer minimizador local de la funcion cuadr\'{a}tica a pedazos $q(x(t))$,
para $t\geq 0$. Este minimizador se obtiene examinando cada uno de los
segmentos de l\'{\i}nea que forman $x(t)$. Para realizar esta b\'{u}squeda,
se determinan los valores $t$ para los cuales ocurren los dobleces o
quiebres. Primero se identifica los valores de $t$ para los cuales cada
componente alcanza su l\'{\i}mite a lo largo de la direcci\'{o}n -$g$. Estos
valores $\hat{t}_{i}$ se obtienen por las siguientes f\'{o}rmulas explicitas:

\bigskip 

$\hat{t}_{i}=\left\{ 
\begin{array}{cc}
(x_{i}^{0}-u_{i})/g_{i} & \text{si }g_{i}<0\text{ y }u_{i}<\infty  \\ 
(x_{i}^{0}-l_{i})/g_{i} & \text{si }g_{i}>0\text{ y }u_{i}>-\infty  \\ 
\infty  & \text{si no ocurre lo anterior}%
\end{array}%
\right. \qquad \qquad (16.46)$

\bigskip 

Los componentes de $x(t)$ para cualquer $t$ son:

\bigskip 

$x_{i}(t)=\left\{ 
\begin{array}{cc}
x_{i}^{0}-tg_{i} & \text{si }t\leq \hat{t}_{i} \\ 
x_{i}^{0}-\hat{t}g_{i} & \text{de lo contrario}%
\end{array}%
\right. $

\bigskip 

Para buscar el primer minimizador local a lo largo de $P(x^{0}-tg,l,u)$, se
eliminan los valors duplicados y cero de $\hat{t}_{i}$ del conjunto \{$\hat{t%
}_{1},\hat{t}_{2},...,\hat{t}_{n}$\} y se ordenan los elementos restantes en
una secuencia ordenada $t_{1},t_{2},...$ tal que $0\leq t_{1}\leq t_{2}...,$
luego se examinan los int\'{e}rvalos $[0,t_{1}],[t_{1},t_{2}],...$ en turno.
Suponiendo que se han examinado los int\'{e}rvalos hasta $[t_{j-2},t_{j-1}]$
para alg\'{u}n $j$, y determinado que el minimizador local esta en alg\'{u}n
valor $t\geq t_{j-1}$. Para el int\'{e}rvalo $[t_{j-1},t_{j}]$ entre el ($j-1
$)avo punto de quiebre, tenemos que

\bigskip 

$x(t)=x(t_{j-1})+\bigtriangleup tp^{j-1}$

\bigskip 

donde

\bigskip 

$\bigtriangleup t=t-t_{j-1},$ \ \ \ \ \ $\bigtriangleup t\in \lbrack
0,t_{j}-t_{j-1}],$

\bigskip 

y

\bigskip 

$p^{j-1}=\left\{ 
\begin{array}{cc}
-g_{i} & \text{si }t_{j-1}\,\hat{t}_{i,} \\ 
0 & \text{de lo contrario}%
\end{array}%
\right. \qquad \qquad \qquad (16.47)$

\bigskip 

Usando esta notaci\'{o}n, escribimos la cuadr\'{a}tica (16.43a) en el
segmento de l\'{\i}nea $[x(t_{j-1}),x(t_{j})]$ como

\bigskip 

$q(x(t))=d^{T}(x(t_{j-1})+\Delta tp^{j-1})+\frac{1}{2}(x(t_{j-1})+\Delta
tp^{j-1})^{T}G(x(t_{j-1})+\Delta tp^{j-1})$,

\bigskip 

donde $\Delta t\in \lbrack 0,t_{j}-t_{j-1}]$.

\bigskip 

Expandiendo y agrupando los coeficientes de $1$, $\Delta t$, y $(\Delta
t)^{2}$, encontramos que

\bigskip 

$q(x(t))=f_{j-1}+f_{j-1}^{\prime }+\frac{1}{2}f_{j-1}^{\prime \prime
}(\Delta t)^{2}$, \ \ \ \ \ \ $\Delta t\in \lbrack 0,t_{j}-t_{j-1}]$, \ \ \
\ \ \ \ \ \ \ (16.48)

\bigskip 

donde los coeficientes $f_{j-1},f_{j-1}^{\prime },$ y $f_{j-1}^{\prime
\prime }$ se definen como

\bigskip 

\begin{eqnarray*}
f_{j-1} &=&d^{T}x(t_{j-1})+\frac{1}{2}x(t_{j-1})^{T}Gx(t_{j-1}), \\
f_{j-1}^{\prime } &=&d^{T}p^{j-1}+x(t_{j-1})^{T}Gp^{j-1}, \\
f_{j-1}^{\prime \prime } &=&(p^{j-1})^{T}Gp^{j-1}.
\end{eqnarray*}

\bigskip 

diferenciando (16.48) e igualando a cero

\bigskip 

$\Delta t^{\ast }=-f_{j-1}/f_{j-1}^{\prime \prime }$

\bigskip 

si $\Delta t^{\ast }\in \lbrack 0,t_{j}-t_{j-1}]$ tenemos el primer m\'{\i}%
nimo local $x^{c}$. Si $\Delta t^{\ast }$ no cae en el int\'{e}rvalo, se
busca en el int\'{e}rvalo siguiente $[t_{j},t_{j+1}]$ hasta encontrar un $%
x^{c}$.

\bigskip 

Ahora se contin\'{u}a con el segundo paso del m\'{e}todo. Dado $x^{c}$ y su
conjunto activo $A(x^{c})$ se resuelve aproximadamente el siguiente problema
de programaci\'{o}n cuadr\'{a}tica:

\bigskip 

\begin{eqnarray*}
\min_{x}q(x) &=&\frac{1}{2}x^{T}Gx+x^{T}d\text{ \ \ \ \ \ \ \ \ \ \ \ \ \ \
\ \ \ \ (16.49a)} \\
\text{sujeto a }x_{i} &=&x_{i}^{c}\text{, \ \ \ \ }i\in A(x^{c})\text{ \ \ \
\ \ \ \ \ \ \ \ \ \ \ \ \ (16.49b)} \\
l_{i} &\leq &x_{i}\leq u_{i},\text{ \ \ \ }i\notin A(x^{c})\text{ \ \ \ \ \
\ \ \ (16.49c)}
\end{eqnarray*}

\bigskip 

A continuaci\'{o}n se presenta el m\'{e}todo de proyecci\'{o}n de gradiente
para programaci\'{o}n cuadr\'{a}tica.

\bigskip 

Calcular un punto inicial factible $x^{0}$;

\textbf{para} $k=0,1,2,...$

\qquad \textbf{si} $x^{k}$ satisface las condiciones KKT para (16.43)

\qquad \qquad PARAR con la soluci\'{o}n $x^{\ast }=x^{k}$;

\qquad Hacer $x=x^{k}$ y encontrar el punto de Cauchy $x^{c}$;

\qquad Encontrar una soluci\'{o}n aproximada $x^{+}$ de (16.49) tal que $%
q(x^{+})<q(x^{c})$ y $x^{+}$ es factible;

\qquad $x^{k+1}\leftarrow x^{+};$

\textbf{fin} (\textbf{para})

\bigskip 

M\'{E}TODOS DE PUNTO INTERIOR

\bigskip 

Se pueden usar los m\'{e}todos de punto interior primales duales para
resolver los problemas de progrmaci\'{o}n cuadr\'{a}tica. Se considerar\'{a}
el caso convexo ($G$ es semipositiva definida). Planteamos el problema de la
siguiente forma:

\bigskip 

\begin{eqnarray*}
\min_{x}q(x) &=&\frac{1}{2}x^{T}Gx+d^{T}x\text{ \ \ \ \ \ (16.50a)} \\
\text{sujeto a }Ax &\geq &b\text{ \ \ \ \ \ \ \ \ \ \ \ \ \ \ \ \ \ \ \ \ \
\ \ \ \ \ (16.50b)}
\end{eqnarray*}

\bigskip 

\bigskip $G$ es positiva definida, $A$ es una mariz de $m$ x $n$ donde $m$ $%
\leq n$, $x\in R^{n}$, $b\in R^{m}$.

\bigskip 

Si $x^{\ast }$ es soluci\'{o}n del problema, entonces existe un vector $%
\lambda $ tal que se satisfacen las condiciones de KKT. Calculando el
Lagragiano e igualando a cero

\bigskip 

\begin{eqnarray*}
Gx+d-A\lambda  &=&0, \\
Ax-b &\geq &0, \\
\lambda ,x &\geq &0
\end{eqnarray*}%
$\bigskip $

\bigskip 

Introduciendo variables de holgura $y=Ax-b$.

\bigskip 

\begin{eqnarray*}
Gx-A^{T}\lambda +d &=&0, \\
Ax-y-b &=&0, \\
y_{i}\lambda _{i} &=&0, \\
(\lambda ,y) &\geq &0
\end{eqnarray*}

\bigskip 

El problema obtenido equivale a resolver:

\bigskip 

$F(x,y,\lambda )=\left\{ 
\begin{array}{c}
Gx-A^{T}y+d \\ 
Ax-y-b \\ 
Y\Lambda e%
\end{array}%
=0\right. $

\bigskip 

sujeto a $(y,\lambda )\geq 0$

\bigskip 

$Y=diag(y_{1},y_{2},..,y_{m})$

$\Lambda =diag(\lambda _{1},\lambda _{2},..,\lambda _{m})$

\bigskip 

Usando el m\'{e}todo del camino central, el sistema KKT queda:

\bigskip 

$\left[ 
\begin{array}{ccc}
G & -A^{T} & 0 \\ 
A & -I & 0 \\ 
0 & \Lambda  & Y%
\end{array}%
\right] \left[ 
\begin{array}{c}
\Delta x \\ 
\Delta y \\ 
\Delta \lambda 
\end{array}%
\right] =\left[ 
\begin{array}{c}
-rd \\ 
-rb \\ 
-Y\Lambda e+\sigma \mu e%
\end{array}%
\right] $

\bigskip 

donde 

\begin{eqnarray*}
rd &=&Gx-A^{T}\lambda +d, \\
rb &=&Ax-y-b, \\
(y,\lambda ) &\geq &0, \\
\mu  &=&\frac{1}{m}\sum_{i=1}^{m}y_{i}\lambda _{i}, \\
\sigma  &\in &[0,1]
\end{eqnarray*}

\end{document}
