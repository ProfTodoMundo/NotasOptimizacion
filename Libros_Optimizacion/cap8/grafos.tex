\documentclass{report}
%%%%%%%%%%%%%%%%%%%%%%%%%%%%%%%%%%%%%%%%%%%%%%%%%%%%%%%%%%%%%%%%%%%%%%%%%%%%%%%%%%%%%%%%%%%%%%%%%%%%%%%%%%%%%%%%%%%%%%%%%%%%
\usepackage{amsfonts}
\usepackage{graphicx}
\usepackage[spanish]{babel}
%\usepackage{latexsym}
%\parskip 1ex
%\linespread{1.3}%
%\oddsidemargin 0.2in%
%\textwidth 6in%
%\topmargin 0in%
%\textheight 8.5in
%\parindent 2em
\decimalpoint

\begin{document}

\title{Cap\'itulo $\#8$ de Optimizaci\'on II}

\author{Angel Eduardo Mu\~noz Zavala}
%\institute{Center for Research in Mathematics (CIMAT)\\
%Department of Computer Science\\
%A.P. 402, Guanajuato, Gto. 36240, M\'{e}xico\\
%\email{aemz,artha@cimat.mx}}

\maketitle


%\begin{keywords}
%Constrained Optimization, Evolutionary computation, Particle swarm
%optimization.
%\end{keywords}



\section{El camino m\'as corto}
El camino m\'as corto de un nodo $i$ a un nodo $j$, en un grafo dirigido o no dirigido, es un problema de combinatoria; ya que existe un n\'umero finito de posibilidades para viajar del nodo $i$ al nodo $j$, pasando a lo m\'as una vez por el mismo nodo.\bigskip

Como mencione anteriormente, el an\'alisis es aplicable a grafos dirigidos o no dirigidos; imaginemos que tenemos el siguiente grafo dirigido que se muestra en la Figura \ref{fig:grafo}:
  
\begin{figure}[hb]
    \centering
    \includegraphics[width=6cm]{grafo.eps}
    \caption{Grafo Dirigido}
    \label{fig:grafo}
\end{figure}

Es un ejemplo muy sencillo, donde a simple vista podemos obtener el camino m\'as corto de un nodo a otro. Por ejemplo, el camino mas corto del nodo $0$ al nodo $7$, posee una distancia de $5$ unidades y 3 caminos diferentes con esa misma distancia; como s eilustra en la Figura \ref{fig:grafo}.\bigskip

Desafortunadamente, no todos los grafos son asi de sencillos, pueden tener un enorme n\'umero de nodos y a\'un m\'as aristas. Por este motivo, existen algoritmos que nos sirven para resolver nuestro problema de una forma iterativa. Pero antes de ver uno de esos algoritmos, primero debemos realizar ciertas definiciones.



\section{Conceptos B\'asicos}
Primero definamos un grafo $D$, como el conjunto de nodos $V$ y de aristas $A$; $D=\{V,A\}$. Y sea $C_{i,j}$ el costo de viajar del nodo $i$ al nodo $j$; es decir, la distancia entre 2 nodos contenidos en $V$.\bigskip

Entonces, el costo total de ir de un nodo $X$ a un nodo $Y$, es la suma de todas las aristas visitadas; es decir:\bigskip

\noindent Sea $P(V_i,V_j)$ el camino a seguir para viajar del nodo $V_i$ al nodo $V_j$, que pertencen al conjunto de nodos $V$: 

\begin{eqnarray}
	P(V_i,V_j) & = & \left\{ w_0,w_1,...,w_n : w_0=V_i, w_n=V_j,\right.\nonumber\\ 
	& & \left. (w_k,w_{k+1})\in A, 0 \leq k < n \right\}
\end{eqnarray}\bigskip

\noindent El costo de un camino $P(V_i,V_j)$ se define como:

\begin{eqnarray}
	C(P(V_i,V_j)) = \sum_{k=0}^{n-1} C_{w_k,w_{k+1}} \qquad \forall w_k \in P(V_i,V_j)
\end{eqnarray}\bigskip

El problema de encontrar el camino m\'as corto del nodo $V_i$ al nodo $V_j$, se puede plantear como:

\begin{eqnarray}
	\min_{P(V_i,V_j)} C(P(V_i,V_j))
\end{eqnarray}\bigskip

Una forma de atacar el problema es:

\begin{eqnarray}
	\min_{P(V_i,X)} C(P(V_i,X)) \qquad \forall X \in V
\end{eqnarray}\bigskip

Por lo tanto, equivale a encontrar los caminos m\'as cortos del nodo $V_i$ a cualquier nodo $X \in V$. Y para este problema de combinatoria usaremos el algoritmo que se revisa en la siguiente secci\'on.



\section{M\'etodo Dijkstra}
El algoritmo de Dijkstra, se utiliza para encontrar la distancia m\'as corta de un nodo $V_i$ a otro nodo $X$ que pertenecen a $V$, resolviendo una seria de subproblemas o subgrafos que van encontrando las distancias m\'as cortas entre $V_i$ y $Y \in W$, donde $W_k$ es el conjunto de nodos del subgrafo en la iteraci\'on k-\'esima.\bigskip

El algoritmo de Dijkstra se presenta en la Figura \ref{fig:dijkstra}, donde $V$ es el conjunto de nodos y $A$ el conjunto de aristas que pertenecen al grafo $D$.

\begin{figure}[htb]
    \small
    \centering
    \begin{tabular}{|p{8cm}|}
    \hline
		Dado $D=\{V,A\}$ con costos $C_{i,j}\geq 0$ $\forall (i,j)\in A$ \\
		Y dado el nodo $s\in V$\\ \\ 
		Inicializar $W_0=\{s\}, \rho(s)=0$\\ \\
		$\forall y \in V-W_0$ \\ 
\begin{eqnarray}
	\rho(y)=\left\{
	\begin{array}{ccc}
	C_{s,y} & si & (s,y)\in A\\
	\infty & si & (s,y)\notin A\\
	\end{array}
	\right.\nonumber
\end{eqnarray}\bigskip\\
		While $W_k\neq V$\\\\
		\qquad Encontrar $\rho(x)=\min\{\rho(y): y\notin W\}$\\\\
		\qquad Hacer $W_{k+1}=W_k \cup \{x\}$\\\\
		\qquad $\forall y \in V-W_{k+1}$ \\\\
		\qquad \qquad $\rho(y)=\min\{\rho(y),\rho(x)+C_{x,y}\}$\\
\begin{eqnarray}
	C_{x,y}=\left\{
	\begin{array}{ccc}
	C_{x,y} & si & (x,y)\in A\\
	\infty & si & (x,y)\notin A\\
	\end{array}
	\right.\nonumber
\end{eqnarray}\bigskip
    End\\
    \hline
    \end{tabular}
    \normalsize
    \caption{Pseudoc\'odigo del M\'etodo Dijkstra}
    \label{fig:dijkstra}
\end{figure}

\noindent donde $\rho(z)$ es la distancia del camino m\'as corto del nodo $s$ al nodo $z$. Pero, este m\'etodo no tiene el procedimiento necesario para conocer el camino que se siguio para llegar de $s$ a cada nodo; que es en especial el tema de este reporte. Para conocer el camino que el m\'etodo de Dijkstra siguio para encontrar la distancia minima de un nodo $s$ a un nodo $y$.\bigskip

Notemos que en cualquier instante de la propagaci\'on existe un conjunto $W$ de nodos con costo $\rho(x) \forall x \in V$. Por \'ultimo, el camino m\'as corto de $s$ a $y$ utiliza utiliza unicamente nodos en $W$, de otra forma existir\'ia un $z \notin V$ cuya $\rho(z)<\rho(y)$ y tendr\'ia que insertarse primero en $W$, para despu\'es llegar a $y$. 
\end{document}


