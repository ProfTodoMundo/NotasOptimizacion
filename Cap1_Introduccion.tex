% !TEX root = NotasCursoOptimizacion.tex

%===========================================
\section{Contenido del curso}
%===========================================

Los temas a cubrir en el curso son los siguientes:

\begin{itemize}
\item[I ] Formulaci\'on matem\'atica de un problema de Optimizaci\'on con restricciones

\item[II ] Condiciones de optimalidad de primero y segundo orden

\item[III ] Interpretaci\'on geom\'etrica

\item[IV ] Programaci\'on lineal: el m\'etodo simplex y el de puntos interiores

\item[V ] Programaci\'on cuadr\'atica

\end{itemize}


El prop\'osito general del curso es que \textbf{el/la estudiante comprenda y aplique la teor\'ia, t\'ecnicas y m\'etodos para resolver problemas de optimizaci\'on con restricciones}.

\textbf{Metodolog\'ia del curso}: el curso consistir\'a en 3 sesiones a la semana en las que se desarrollar\'an y revisar\'an los contenidos del curso, las sesiones ser\'an te\'oricas y se complementar\'an con pr\'acticas en \texttt{R}, a trav\'es de scripts en formato \texttt{Rmd} los cuales se compilar\'an en un \'unico documento al que se le denominar\'a \texttt{Portafolio}. Adem\'as del estudio de los contenidos del curso, el/la estudiante desarrollar\'a a lo largo del curso $4$ biograf\'ias de una lista de $52$ cient\'ificos prominentes en la historia de la ciencia. La evaluaci\'on de los contenidos del curso ser\'a a trav\'es de tres evaluaciones formativas en las cuales se deber\'a de demostrar el dominio de los temas revisados en clase. La asistencia ser\'a importante para poder ser evaluado con estos tres elementos, el requisito de porcentaje de asistencia es del $85\%$, en caso de no contar con este porcentaje m\'inimo de asistencia para poder certificar la materia deber\'a de presentar el examen de certificaci\'on, mismo que ser\'a elaborado por el comit\'e de certificaci\'on.


\textbf{Evaluaci\'on del curso}: La calificaci\'on final del curso se obtiene de la suma de los siguientes porcentajes: \textit{Calificaci\'on Final} $=$ \textit{Portafolio} (30\%) $+$ \textit{Biograf\'ias} ($10\%$) $ +$ \textit{Evaluaciones} ($60\%$).
